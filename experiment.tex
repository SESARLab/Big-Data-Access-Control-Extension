\section{Experiments}\label{sec:experiment}
In this section, we will discuss the experiments conducted to evaluate the effectiveness, particularly of the proposed metrics and framework as a whole. The evaluation process proceeded as follows:

\begin{enumerate}
  \item Generation of synthetic datasets from a public dataset: Synthetic datasets were generated with the aim of testing the performance of the metrics. These datasets were preceded by perturbations designed to alter both the number of items using a purely quantitative approach and the statistical distribution of the dataset. The alterations involved multiple columns to assess the behavior of the weighted metrics.
  \item Identification of parameters: To apply the modifications mentioned above, it was necessary to identify the parameters to be altered. Figure \ref{fig:distributions} shows plots of the statistical distributions and features involved in the identification process.
  \item Implementation of software environment and tools: The following components were implemented: a software environment aimed at emulating the behaviors of the services, the heuristics, and a tool to emulate the behavior of the framework as a whole. This tool enabled the selection of services from a predetermined set, with the goal of minimizing the metrics by applying the proposed heuristics.
  \item Comparison with manually computed optimum: Finally, the results of the end-to-end test were compared with the manually computed optimum. Table \ref{tab:results} presents the compared and ranked results obtained from this comparison.
\end{enumerate}

By following this experimental procedure, the effectiveness of the proposed metrics and framework was thoroughly evaluated, providing valuable insights into their performance and capabilities.

\begin{figure}[ht]
  \centering
  \includegraphics[width=\columnwidth]{example-image-a}
  \caption{Plots of the statistical distributions and features involved in the identification of parameters.}
  \label{fig:distributions}
\end{figure}

% \begin{table}[ht]
%   \centering
%   \caption{Comparison and ranking of the results obtained from the end-to-end test and the manually computed optimum.}
%   \label{tab:results}
%   \begin{tabular}{c|c}
%     \textbf{Rank} & \textbf{Result Comparison} \

%     $\vdots$      & $\vdots$ \
%   \end{tabular}
% \end{table}

This experimental approach provides a assessment of the proposed metrics and framework, offering insights into their performance and capabilities.