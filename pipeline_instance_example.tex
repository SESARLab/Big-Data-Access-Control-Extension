\begin{example}\label{ex:instance}

  Let us consider a subset \{\vi{6}, \vi{7}, \vi{8}\} of the pipeline template \tChartFunction in \cref{sec:example}.

  Each vertex is associated with three candidate services, each having a profile. The filtering algorithm matches each candidate service's profile with the policies annotating the corresponding vertex. It returns the set of services whose profile matches a policy.
  \begin{enumerate*}[label=\textit{\roman*})]
    \item For \vi{6}, the filtering algorithm produces the set $S'_1=\{s_{11},s_{12}\}$; assuming that the dataset owner is ``CT'', the service profile of \s{11} matches \p{1} and the one of \s{12} matches \p{2}.
          For \s{13}, there is no policy match and, thus, it is discarded.
    \item For \vi{7}, the filtering algorithm returns the set $S'_2=\{s_{22},s_{23}\}$; assuming that the dataset region is ``CT'', the service profile of \s{22} matches \p{5} and the one of \s{23} matches \p{6}.
          For \s{21}, there is no policy match and, thus, it is discarded.
    \item For \vi{8}, the filtering algorithm returns the set $S'_3=\{s_{31},s_{32},s_{33}\}$. Since policy \p{7} matches with any subjects, the filtering algorithm does not discard any service.
  \end{enumerate*}

  The comparison algorithm is then applied to the set of services $S'_*$ and it returns a ranking of the services.
  The ranking is based on the amount of data that is anonymized by the service.
  The ranking is listed in \cref{tab:instance_example} and it is based on the transformation function of the policies,
  assuming that a more restrictive transformation function anonymizes more data affecting negatively the position in the ranking.
  For example, \s{11} is ranked first because it anonymizes less data than \s{12} and \s{13}.
  The ranking of \s{22} and \s{23} is based on the same logic.
  Finally, the ranking of \s{31}, \s{32} is influenced by the environment state at the time of the ranking.
  For example, if the environment in which the visualization is performed is a CT facility, then \s{31} is ranked first and \s{32} second;
  thus because the facility is considered a less risky environment than the cloud.



  \begin{table*}
    \def\arraystretch{1.5}
    \caption{Instance example}\label{tab:instance_example}

    \centering
    \begin{tabular}{l|l|c|c|c}

      \textbf{Vertex$\rightarrow$Policy}                   & \textbf{Candidate} & \textbf{Profile}                         & \textbf{Filtering} & \textbf{Ranking} \\
      \multirow{ 3}{*}{\vi{4}  $\rightarrow$ \p{1},\p{2} } & $\s{11}$           & service\_owner =    "CT"                 & \cmark             & 1                \\
                                                           & $\s{12}$           & service\_owner =    "NY"                 & \cmark             & 2                \\
                                                           & $\s{13}$           & service\_owner =    "CA"                 & \xmark             & 3                \\
      \hline
      \multirow{ 3}{*}{\vi{7}  $\rightarrow$ \p{5},\p{6} } & $\s{21}$           & service\_region =    "CA"                & \xmark             & --               \\
                                                           & $\s{22}$           & service\_region =    "CT"                & \cmark             & 1                \\
                                                           & $\s{23}$           & service\_region =    "NY"                & \cmark             & 2                \\
      \hline
      \multirow{ 3}{*}{\vi{8}  $\rightarrow$ \p{7},\p{8} } & $\s{31}$           & visualization\_location = "CT\_FACILITY" & \cmark             & 1                \\
                                                           & $\s{32}$           & visualization\_location = "CLOUD"        & \cmark             & 2                \\
      \hline
    \end{tabular}
  \end{table*}
  \begin{table*}[htbp]
    \centering

    \caption{A test caption}
    \label{table2}
  \end{table*}



\end{example}

\subsection{NP-Hardness of the Pipeline Instantiation Process}
We need to\\
1 - define the quality metric\\
2 - define the problem as a max-instance problem, cioè la definizione di un'istanza con max quality\\
3 - definiamo che + NP-Hard

\begin{problem}

\end{problem}

Note that while the overall problem is NP-hard, there is a component of the problem that is solvable in polynomial time: matching the profile of each service with the node policy.
This can be done by iterating over each node and each service, checking if the service matches the node’s policy.
This process would take $O(|N|*|S|)$ time. This is polynomial time complexity.

The \problem is NP-hard, as stated by the following theorem
\begin{theorem}
  The \problem is NP-Hard
\end{theorem}
\emph{Proof: }

The proof is a reduction from the NP-hard problem. We map each service s in S to an item in the Knapsack Problem.
The value of the item is equivalent to the calculated metric, and the weight of the item is uniformly 1, as we can choose each service once. The capacity of the knapsack is set to the number of nodes.
Our problem can now be viewed as a variant of the Knapsack Problem: find the subset of items(services)
that maximizes the total value (score) without exceeding the capacity of the knapsack (number of nodes).
The Knapsack Problem is NP-hard.
Since our problem can be reduced to the Knapsack Problem in polynomial time, our problem is also NP-hard.


% The metrics established will enable the quantification of data loss pre- and post-transformations.
% In the event of multiple service interactions, each with its respective transformation,
% efforts will be made to minimize the loss of information while upholding privacy and security standards.
% Due to the exponential increase in complexity as the number of services and transformations grow,
% identifying the optimal path is inherently an NP-hard problem.
% As such, we propose some heuristics to approximate the optimal path as closely as possible.
%To evaluate their efficacy, the heuristically generated paths will be compared against the optimal solution.