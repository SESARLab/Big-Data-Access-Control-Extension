\subsection{Example}\label{sec:example}
\newcommand{\pone}{$\langle service\_owner=dataset\_owner\rangle$}
\newcommand{\ptwo}{$\langle service\_owner=partner(dataset\_owner) \rangle$}
\newcommand{\pthree}{$\langle service\_owner \neq dataset\_owner AND owner \neq partner(dataset\_owner)$}

\begin{table*}[ht!]
  \def\arraystretch{1.5}
  \centering
  \caption{Anonymization policies (a) and data transformations (b)}\label{tab:anonymization}

  \resizebox{\textwidth}{!}{
    \begin{tabular}[t]{cc}
      \begin{tabular}[t]{c|c|l}
        \textbf{Vertex}      & \textbf{Policy} & \policy{subject}{object}{action}{environment}{transformation}                           \\ \hline
        \vi{1},\vi{2},\vi{3} & $\p{0}$         & \policy{ANY}{dataset}{READ}{ANY}{\tp{0}}                                                \\
        \vi{4},\vi{5}        & $\p{1}$         & \policy{\pone}{dataset}{READ}{ANY}{\tp{0}}                                              \\
        \vi{4},\vi{5}        & $\p{2}$         & \policy{\ptwo}{dataset}{READ}{ANY}{\tp{1}}                                              \\
        %\vi{4},\vi{6}        & $\p{3}$         & \policy{\pthree}{dataset}{READ}{ANY}{\tp{2}}                                                    \\
        \vi{6}               & $\p{5}$         & \policy{$\langle service\_region= dataset\_origin\rangle$}{dataset}{WRITE}{ANY}{\tp{0}} \\
        \vi{6}               & $\p{6}$         & \policy{$\langle service\_region=``\{NY,NH\}"\rangle$}{dataset}{WRITE}{ANY}{\tp{1}}     \\
        \vi{7}               & $\p{7}$         & \policy{ANY}{dataset} {READ}{$environment = risky$}{\tp{3}}                             \\
        \vi{7}               & $\p{8}$         & \policy{ANY}{dataset} {READ}{$environment = not\_risky$}{\tp{4}}                        \\
      \end{tabular}
                        &

      %\caption{Anonymization levels}\label{tab:levels}
      \begin{tabular}[t]{c|c|l}
        \textbf{\tp{i}} & \textbf{Level} & \textbf{Data Transformation}                      \\\hline
        \tp{0}          & $l_0$          & $anon(\varnothing)$                               \\
        \tp{1}          & $l_1$          & $anon(fname, lname)$                              \\
        \tp{2}          & $l_2$          & $anon(fname, lname, id, age)$                     \\
        \tp{3}          & $r_0$          & $aggregation(cluster=\infty)                    $ \\
        \tp{4}          & $r_1$          & $aggregation(cluster=10)                       $  \\
        \multicolumn{3}{c}{\footnotesize (b)}
      \end{tabular} \\
      \footnotesize (a) &                                                                                          \\
    \end{tabular}
  }
\end{table*}
Let us consider our reference scenario in \cref{sec:systemmodel}.
\cref{fig:service_composition_template} presents an example of pipeline template consisting of five stages, each one annotated with a policy in \cref{tab:anonymization}.
% We recall that \cref{tab:dataset} shows a sample of our reference dataset.

% 1° NODO %
The first stage consists of three parallel vertices \vi{1}, \vi{2}, \vi{3} for data collection.
Data protection annotations \myLambda(\vi{1}), \myLambda(\vi{2}), \myLambda(\vi{3}) refer to policy \p{0} with an empty transformation \tp{0}.
Functional requirement \F{1}, \F{2}, \F{3}  prescribes a URI as input and the corresponding dataset as output.

The second stage consists of vertex \vi{4},
merging the three datasets obtained stage 1. Data protection annotation \myLambda(\vi{4}) refers to policies \p{1} and \p{2}, which apply different data transformations depending on the relation between the dataset and service owners.
% 2° NODO %
If the service owner is also the dataset owner (\pone), the dataset is not anonymized (\tp{0}). We note that if the service owner has no partner relationship with the dataset owner, no policies apply.
If the service owner is a partner of the dataset owner (\ptwo), the dataset is anonymized at level $l_1$ (\tp{1}).
%if the service owner is neither the dataset owner nor a partner of the dataset owner  (\pthree), the dataset is anonymized level2 (\tp{2}).
Functional requirement \F{4} prescribes $n$ datasets as input and the merged dataset as output.

% 3° NODO %
The third stage consists of vertex \vi{5}  for data analysis.
Data protection annotation \myLambda(\vi{5}) refers to policies \p{1} and \p{2}, as for stage 2.
% The logic remains consistent:
% if the service profile matches with the data owner (\pone), \p{1} matches and level0 anonymization is applied (\tp{0});
% if the service profile matches with a partner of the owner (\ptwo), \p{2} matches and level1 anonymization is applied  (\tp{1});
% if the service profile doesn't match with a partner nor with the owner (\pthree), \p{3} matches and level2 anonymization is applied (\tp{2}).
Functional requirement \F{5} prescribes a dataset as input and the results of the data analysis as output.
% 4° NODO %
Data protection annotation \myLambda(\vi{5}) refers to policy \p{4} with data transformation \tp{2}, that is, anonymization level $l_2$ to prevent personal identifiers from entering into the machine learning algorithm/model.
Functional requirement \F{6} prescribes a dataset as input, and the trained model and a set of inferences as output.

% 5° NODO %
The fourth stage consists of vertex \vi{6}, managing data storage. Data protection annotation \myLambda(\vi{6}) refers to policies \p{5} and \p{6},
which apply different data transformations depending on the relation between the dataset and service region.
If the service region is the dataset origin ($\langle service\_region=dataset\_origin"\rangle$) (\p{5}), the dataset is anonymized at level $l_1$ (\tp{1}).
If the service region is in a partner region ($\langle service,region={NY,NH}"\rangle$) (\p{6}), the dataset is anonymized at level $l_2$ (\tp{2}).
Functional requirement \F{7} prescribes a dataset as input and the URI of the stored data as output.

% 6° NODO %
The sixth and last stage consists of vertex \vi{7}, responsible for data visualization.
Data protection annotation \myLambda(\vi{7}) refers to policies \p{7} and \p{8}, which anonymize data according to the environment where the service is executed.
A \emph{risky} environment is defined as a region outside the owner or partner facility.
If the environment is risky (\p{7}), the data are anonymized at level $r_0$ (\tp{3}).
If the environment is not risky (\p{8}), the data are anonymized at level $r_1$ (\tp{4}).
Functional requirement \F{8} prescribes a dataset as input and data visualization interface (possibly in the form of JSON file) as output.

%In summary, this tion has delineated a comprehensive pipeline template. This illustrative pipeline serves as a blueprint, highlighting the role of policy implementation in safeguarding data protection across diverse operational stages.
