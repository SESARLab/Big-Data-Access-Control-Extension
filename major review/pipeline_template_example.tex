\begin{table*}[!t]
  \def\arraystretch{1.5}
  \centering
  \caption{Anonymization policies (a) and data transformations (b)}\label{tab:anonymization}

  \resizebox{\textwidth}{!}{
    \begin{tabular}[t]{cc}
      \begin{tabular}[t]{c|c|l}
        \textbf{Vertex}      & \textbf{Policy} & \policy{subject}{object}{action}{environment}{transformation}                           \\ \hline
        \vi{1},\vi{2},\vi{3} & $\p{0}$         & \policy{ANY}{dataset}{READ}{ANY}{\tp{0}}                                                \\
        \vi{4},\vi{5}        & $\p{1}$         & \policy{\{\pone\}}{dataset}{READ}{ANY}{\tp{0}}                                              \\
        \vi{4},\vi{5}        & $\p{2}$         & \policy{\{\ptwo\}}{dataset}{READ}{ANY}{\tp{1}}                                              \\
        \vi{6}               & $\p{3}$         & \policy{\{$(service\_region=dataset\_origin)$\}}{dataset}{WRITE}{ANY}{\tp{0}} \\
        \vi{6}               & $\p{4}$         & \policy{\{$(service\_region=\{``NY",``NH"\})$\}}{dataset}{WRITE}{ANY}{\tp{1}}     \\
        \vi{7}               & $\p{5}$         & \policy{ANY}{dataset} {READ}{\langle$environment=``risky''$\rangle}{\tp{3}}                             \\
        \vi{7}               & $\p{6}$         & \policy{ANY}{dataset} {READ}{\langle$environment=``not\_risky''$\rangle}{\tp{4}}                        \\
      \end{tabular}
                        &

      \begin{tabular}[t]{c|c|l}
        \textbf{\tp{i}} & \textbf{Level} & \textbf{Data Transformation}                      \\\hline
        \tp{0}          & $l_0$          & $anon(\varnothing)$                               \\
        \tp{1}          & $l_1$          & $anon(fname, lname)$                              \\
        \tp{2}          & $l_2$          & $anon(fname, lname, id, age)$                     \\
        \tp{3}          & $r_0$          & $aggregation(cluster=\infty)                    $ \\
        \tp{4}          & $r_1$          & $aggregation(cluster=10)                       $  \\
        \multicolumn{3}{c}{\footnotesize (b)}
      \end{tabular} \\
      \footnotesize (a) &                                                                                          \\
    \end{tabular}
  }
\end{table*}

\begin{example}[\bf \pipelineTemplate]\label{ex:template}
Let us consider the reference scenario introduced in \cref{sec:systemmodel}.
\cref{fig:service_composition_template} presents an example of pipeline template consisting of five stages, each one annotated with a policy in \cref{tab:anonymization}.
{\color{OurColor}A visual aid to this example is presented in \cref{fig:pipeline_template_example}}.

% 1° NODO %
The first stage consists of three parallel vertices \vi{1}, \vi{2}, \vi{3} for data collection.
Data protection annotations \myLambda(\vi{1}), \myLambda(\vi{2}), \myLambda(\vi{3}) refer to policy \p{0} with an empty transformation \tp{0}.
Functional requirements \F{1}, \F{2}, \F{3}  prescribe a URI as input and the corresponding dataset as output.

The second stage consists of vertex \vi{4}, merging the three datasets obtained at the first stage. Data protection annotation \myLambda(\vi{4}) refers to policies \p{1} and \p{2}, which apply different data transformations depending on the relation between the dataset and the service owner.
% 2° NODO %
If the service owner is also the dataset owner (i.e., \pone), the dataset is not anonymized (\tp{0}). If the service owner is a partner of the dataset owner (i.e., \ptwo), the dataset is anonymized at \emph{level1} (\tp{1}). If the service owner has no partner relationship with the dataset owner, no policy applies.
Functional requirement \F{4} prescribes $n$ datasets as input and the merged dataset as output.

% 3° NODO %
The third stage consists of vertex \vi{5}  for data analysis.
Data protection annotation \myLambda(\vi{5}) refers to policies \p{1} and \p{2}, as for the second stage.
Functional requirement \F{5} prescribes a dataset as input and the results of the data analysis as output.

% 5° NODO %
The fourth stage consists of vertex \vi{6}, managing data storage. Data protection annotation \myLambda(\vi{6}) refers to policies \p{3} and \p{4}, which apply different data transformations depending on the relation between the dataset and the service region.
If the service region is the dataset origin (condition $(service\_region$$=$$dataset\_origin)$ in \p{3}), the dataset is anonymized at level $l_0$ (\tp{0}).
If the service region is in a partner region (condition ($service\_region$=\{``$NY$",``$NH$"\}) in \p{4}), the dataset is anonymized at level $l_1$ (\tp{1}).
Functional requirement \F{7} prescribes a dataset as input and the URI of the stored data as output.

% 6° NODO %
The last stage consists of vertex \vi{7}, responsible for data visualization.
Data protection annotation \myLambda(\vi{7}) refers to policies \p{5} and \p{6}, which anonymize data according to the environment where the service is executed.
A \emph{risky} environment is defined as a region outside the owner or partner facility.
If the environment is risky (\p{5}), the data are anonymized at level $r_0$ (\tp{3}).
If the environment is not risky (\p{6}), the data are anonymized at level $r_1$ (\tp{4}).
Functional requirement \F{8} prescribes a dataset as input and data visualization interface (possibly in the form of JSON file) as output.
\end{example}

\begin{figure}
  \centering
  % \resizebox{\columnwidth}{!}{%
  \begin{tikzpicture}[
    box/.style={draw, circle},
    arrow/.style={->, >=stealth, thick},
    scale=0.9
]
% Nodes for the stages
\node[box] (v1) {$\vi{1}$};
\node[box, below=of v1] (v2) {$\vi{2}$};
\node[box, below=of v2] (v3) {$\vi{3}$};
\node[box, right=of v2, xshift=1cm] (v4) {$\vi{4}$};
\node[box, right=of v4] (v5) {$\vi{5}$};
\node[box, right=of v5] (v6) {$\vi{6}$};
\node[box, right=of v6] (v7) {$\vi{7}$};

\node[above] at (v1.north)  {(\p{0}, \tp{0})};
\node[above] at (v2.north)  {(\p{0}, \tp{0})};
\node[above] at (v3.north)  {(\p{0}, \tp{0})};
\node[above] at (v4.north)  {(\p{1}, \tp{2})};
\node[above] at (v5.north)  {(\p{1}, \tp{2})};
\node[above] at (v6.north)  {(\p{3}, \tp{4})};
\node[above] at (v7.north)  {(\p{5}, \tp{6})};
% Arrows connecting the stages
\draw[arrow] (v1) -- (v4);
\draw[arrow] (v2) -- (v4);
\draw[arrow] (v3) -- (v4);
\draw[arrow] (v4) -- (v5);
\draw[arrow] (v5) -- (v6);
\draw[arrow] (v6) -- (v7);

\end{tikzpicture}
  % }
  \caption{\label{fig:pipeline_template_example}Example of pipeline template}
\end{figure}