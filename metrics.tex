\section{Maximizing the Pipeline Instance Quality}\label{sec:heuristics}
%
% %Ovviamente non è sufficiente scegliere il best service per ogni vertice, ma diventa un problema complesso dove si devono calcolare/valutare tutte le possibili combinazioni dei servizi disponibili, tra le quali scegliere la migliore.
Our goal is to generate a pipeline instance with maximum quality, addressing data protection requirements while minimizing information loss \textit{dloss} throughout the pipeline execution. To this aim, we first discuss the quality metrics used to measure and monitor data quality, which guide the generation of the pipeline instance. Then, we prove that the problem of generating a pipeline instance with maximum quality is NP-hard (\cref{sec:nphard}). Finally, we present a parametric heuristic (\cref{subsec:heuristics}) tailored to address the computational complexity associated with enumerating all possible combinations within a given set. The primary aim of the heuristic is to approximate the optimal path for service interactions and transformations, particularly within the landscape of more complex pipelines composed of numerous vertices and candidate services. Our focus extends beyond identifying optimal combinations to encompass an understanding of the quality changes introduced during the transformation processes.

%Inspired by existing literature, these metrics, categorized as quantitative and statistical, play a pivotal role in quantifying the impact of policy-driven transformations on the original dataset.

\subsection{Quality Metrics}\label{subsec:metrics}
%Ensuring data quality is mandatory to implement data pipelines that provide accurate results and decision-making along the whole pipeline execution. To this aim, we define two metrics evaluating the quality loss introduced by our policy-driven transformation in Section~\cite{ADD} on the input dataset \origdataset at each step of the data pipeline. Our metrics can be classified as \emph{quantitative} and \emph{qualitative}~\cite{ADD}, and compare the input dataset \origdataset\ and dataset \transdataset\ generated by enforcing data protection requirements on \origdataset.
Ensuring data quality is mandatory to implement data pipelines that provide accurate results and decision-making along the whole pipeline execution. Quality metrics evaluate the information loss introduced at each step of the data pipeline, and can be classified as \emph{quantitative} or \emph{qualitative}~\cite{ADD}.
Quantitative metrics monitor the amount of data lost during data transformations to model the quality difference between datasets \origdataset\ and \transdataset.
Qualitative metrics evaluate changes in the properties of datasets \origdataset\ and \transdataset. For instance, qualitative metrics can measure the changes in the statistical distribution of the two datasets.

In this paper, we use two metrics, one quantitative and one qualitative, to compare the input dataset \origdataset\ and dataset \transdataset\ generated by enforcing data protection requirements (i.e., our policy-driven transformation in \cref{sec:instance}) on \origdataset\ at each step of the data pipeline. We note that a complete taxonomy of possible metrics is outside the scope of this paper and will be the target of our future work.

\subsubsection{Quantitative metric}
%We propose a metric that measures the similarity between two datasets, for this purpose, we use the Jaccard coefficient.
We propose a quantitative metric $M_J$ based on the Jaccard coefficient that assesses the similarity between two datasets. The Jaccard coefficient is defined as follows: \[J(X,Y) = \frac{|X \cap Y|}{|X \cup Y|}\]
where X and Y are two datasets of the same size.

The coefficient is calculated by dividing the cardinality of the intersection of two datasets by the cardinality of their union. It ranges from 0 to 1, with 0 indicating no similarity and 1 indicating complete similarity between the datasets. It has several advantages. Unlike other similarity measures, such as Euclidean distance, it is not affected by the magnitude of the values in the dataset. It is suitable for datasets with categorical variables or nominal data, where the values do not have a meaningful numerical interpretation.

Metric $M_J$ extends the Jaccard coefficient with weights that model the importance of each element in the dataset as follows:\[M_J(X,Y) = \frac{\sum_{i=1}^{n}w_i(x_i \cap y_i)}{\sum_{i=1}^{n}w_i(x_i \cup y_i)}\]
where $x_i$$\in$X ($y_i$$\in$Y, resp.) is the $i$-th feature of dataset X (Y, resp.), and $w_i$ the weight modeling the importance of the $i$-th feature.

It is computed by dividing the cardinality of the intersection of two datasets by the cardinality of their union, weighted by the importance of each feature in the datasets providing a more accurate measure of similarity. %Weights prioritize certain elements (e.g., a specific feature) in the datasets.
%The Weighted Jaccard coefficent can then account for element importance and provide a more accurate measure of similarity.

\subsubsection{Qualitative Metric}
%We propose a metric that enables the measurement of the distance of two distributions.
We propose a qualitative metric $M_{JDS}$ based on the Jensen-Shannon Divergence (JSD) that measures the dissimilarity (distance) between the probability distributions of two datasets.

JSD is a symmetrized version of the KL divergence~\cite{Fuglede} and is applicable to a pair of statistical distributions only. It is defined as follows:
\[JSD(X, Y) = \frac{1}{2} \left( KL(X || M)
  + KL(Y || M) \right)\]
%
where X and Y are two distributions of the same size, and M$=$0.5*(X+Y) is the average distribution.
JSD incorporates both the KL divergence from X to M and from Y to M.

To make JSD applicable to datasets, where each feature in the dataset has its own statistical distribution, metric $M_{JDS}$ applies JSD to each column of the dataset. The obtained results are then aggregated using a weighted average, thus enabling the prioritization of important features that can be lost during the policy-driven transformation in \cref{sec:heuristics}, as follows: \[M_{JDS} = \sum_{i=1}^n w_i \cdot \text{JSD}(x_i,y_i)\]
where \(w_i = \frac{n_i}{N}\) represents the weight for the \(i\)-th column, with \(n_i\) being the number of distinct elements in the $i$-th feature and \(N\) the total number of elements in the dataset. Each \(\text{JSD}(x_i,y_i)\) accounts for the Jensen-Shannon Divergence computed for the \(i\)-th feature in datasets X and Y.

$M_{JDS}$ provides a weighted measure of dissimilarity, which is symmetric and accounts for the contribution from both datasets and specific features. It can compare the dissimilarity of the two datasets, providing a symmetric and normalized measure that considers the overall data distributions.


\subsubsection{Information Loss}
%We note that our metrics can be applied either to the entire dataset or to specific features only. The features can be assigned with equal or varying importance, providing a weighted version of the metrics, thus enabling the prioritization of important features that might be possibly lost during the policy-driven transformation in Section~\cite{ADD}.

Metrics $M_J$ and $M_{JDS}$ contribute to the calculation of the information loss \textit{dloss} throughout the pipeline execution. It is calculated as the average \emph{AVG} of the information loss at each vertex \vi{i}$\in$$\V_S$ of the service pipeline $G(V,E)$ as follows.

\begin{definition}[\emph{dloss}]
  Given a metrics M$\in$$\{M_J,M_{JDS}$\}, information loss \textit{dloss} is calculated as 1$-$\emph{AVG}($M_ij$), with $M_{ij}$ the value of the quality metric retrieved at each vertex \vi{i}$\in$$\V_S$ of the service pipeline $G(V,E)$ according to service \si{j}.
\end{definition}

We note that \textit{dloss}$_{ij}$$=$1$-$$M_i$ models the quality loss at vertex \vi{i}$\in$$\V_S$ of the service pipeline $G(V,E)$ for service \si{j}.
%We also note that information loss \textit{dloss} is used to generate the Max-Quality pipeline instance in the remaining of this section.

\subsection{NP-Hardness of the Max-Quality Pipeline Instantiation Problem}\label{sec:nphard}
%\hl{se lo definiamo in maniera formale come il problema di trovare un'istanza valida in accordo alla definizione di istanza tale che non ne esiste una con un loss piu' piccolo?}
The problem of computing a pipeline instance (\cref{def:instance}) with maximum quality (minimum information loss) can be formally defined as follows.

\begin{definition}[Max-Quality Problem]\label{def:MaXQualityInstance}
  Given a pipeline template $G^{\myLambda,\myGamma}$ and a set $S^c$ of candidate services, find a max-quality pipeline instance $G'$ such that:
  \begin{itemize}
    \item $G'$ satisfies conditions in \cref{def:instance},
    \item $\nexists$ a pipeline instance $G''$ that satisfies conditions in \cref{def:instance} and such that information loss \textit{dtloss}($G''$)$<$\textit{dtloss}($G'$), where \textit{dtloss}($\cdot$) is the information loss throughout the pipeline execution.
    %computed after applying the transformation of the policy matching the service selected to instantiate vertex  \vi{i}$\in$$\V_S$, .
  \end{itemize}
\end{definition}

The Max Quality \problem is a combinatorial selection problem and is NP-hard, as stated by Theorem \cref{theorem:NP}. However, while the overall problem is NP-hard, there is a component of the problem that is solvable in polynomial time: matching the profile of each service with the corresponding vertex policy. This can be done by iterating over each vertex and each service, checking if the service matches the vertex policy. This process would take $O(|N|*|S|)$ time. This is polynomial time complexity.

\begin{theorem}\label{theorem:NP}
  The Max-Quality \problem is NP-Hard.
\end{theorem}
\emph{Proof: }
The proof is a reduction from the multiple-choice knapsack problem (MCKP), a classified NP-hard combinatorial optimization problem, which is a generalization of the simple knapsack problem (KP) \cite{}. In the MCKP problem, there are $t$ mutually disjoint classes $N_1,N_2,\ldots,N_t$ of items to pack in some knapsack of capacity $C$, class $N_i$ having size $n_i$. Each item $j$$\in$$N_i$ has a profit $p_{ij}$ and a weight $w_{ij}$; the problem is to choose one item from each class such that the profit sum is maximized without having the weight sum to exceed C.

The MCKP can be reduced to the Max quality \problem in polynomial time, with $N_1,N_2,\ldots,N_t$ corresponding to $S^c_{1}, S^c_{1}, \ldots, S^c_{u},$, $t$$=$$u$ and $n_i$ the size of $S^c_{i}$. The profit $p_{ij}$ of item $j$$\in$$N_i$ corresponds to \textit{dtloss}$_{ij}$ computed for each candidate service $s_j$$\in$$S^c_{i}$, while $w_{ij}$ is uniformly 1 (thus, C is always equal to the cardinality of $V_C$).

Since the reduction can be done in polynomial time, our problem is also NP-hard. (non è sufficiente, bisogna provare che la soluzione di uno e' anche soluzione dell'altro)


\begin{example}[Max-Quality Pipeline Instance]
  Let us start from \cref{ex:instance} and extend it with the comparison algorithm in \cref{sec:instance} built on \emph{dloss}. The comparison algorithm is applied to the set of services $S'_*$ and returns three service rankings one for each vertex \vi{4}, \vi{5}, \vi{6} according to the amount of data anonymized.
  The ranking is listed in \cref{tab:instance_example_maxquality}(b) and based on the transformation function in the policies. We assume that the more restrictive the transformation function (i.e., it anonymizes more data), the lower is the service position in the ranking.
  For example, \s{11} is ranked first because it anonymizes less data than \s{12} and \s{13}.
  The ranking of \s{22} and \s{23} is based on the same logic.
  Finally, the ranking of \s{31} and \s{32} is affected by the environment state at the time of the ranking.   For example, if the environment where the visualization is performed is a CT facility, then \s{31} is ranked first and \s{32} second because the facility is considered less risky than the cloud.
\end{example}

% The metrics established will enable the quantification of data loss pre- and post-transformations.
% In the event of multiple service interactions, each with its respective transformation,
% efforts will be made to minimize the loss of information while upholding privacy and security standards.
% Due to the exponential increase in complexity as the number of services and transformations grow,
% identifying the optimal path is inherently an NP-hard problem.
% As such, we propose some heuristics to approximate the optimal path as closely as possible.
%To evaluate their efficacy, the heuristically generated paths will be compared against the optimal solution.

\subsection{Heuristic}\label{subsec:heuristics}
%The computational challenge posed by the enumeration of all possible combinations within a given set is a well-established NP-hard problem.}
%The exhaustive exploration of such combinations swiftly becomes impractical in terms of computational time and resources, particularly when dealing with the analysis of complex pipelines.
%In response to this computational complexity, the incorporation of heuristic emerges as a strategy to try to efficiently address the problem.
%\hl{HO RIVISTO IL PARAGRAFO VELOCEMENTE GIUSTO PER DARE UN'INDICAZIONE. DOBBIAMO USARE LA FORMALIZZAZIONE E MAGARI FORMALIZZARE ANCHE LO PSEUDOCODICE.}
We design and implement a heuristic algorithm for computing the pipeline instance maximizing data quality. Our heuristic is built on a \emph{sliding window} and aims to minimize information loss according to quality metrics. At each step, a set of vertices in the pipeline template $\tChartFunction$ is selected according to a specific window size w=[i,j], where $i$ and $j$ are the starting and ending depth of window w. Service filtering and selection in \cref{sec:instance} are then executed to minimize information loss in window w. The heuristic returns as output the list of services instantiating vertexes at depth $i$. A new window w=[i+1,j+1] is considered until $j$+1 is equal to the max depth of $\tChartFunction$, that is, the window reaches the end of the template.
%For example, in our service selection problem where the quantity of information lost needs to be minimized, the sliding window algorithm can be used to select services composition that have the lowest information loss within a fixed-size window.
This strategy ensures that only services with low information loss are selected at each step, minimizing the average information loss. Pseudo-code for the sliding window algorithm is presented in \cref{lst:slidingwindowfirstservice}.

\lstset{ %
  backgroundcolor=\color{white},   % choose the background color; you must add \usepackage{color} or \usepackage{xcolor}
  basicstyle=\footnotesize,        % the size of the fonts that are used for the code
  breakatwhitespace=false,         % sets if automatic breaks should only happen at whitespace
  breaklines=true,                 % sets automatic line breaking
  captionpos=b,                    % sets the caption-position to bottom
  commentstyle=\color{commentsColor}\textit,    % comment style
  deletekeywords={list},            % if you want to delete keywords from the given language
  escapeinside={\%*}{*)},          % if you want to add LaTeX within your code
  extendedchars=true,              % lets you use non-ASCII characters; for 8-bits encodings only, does not work with UTF-8
  frame=tb,	                   	   % adds a frame around the code
  keepspaces=true,                 % keeps spaces in text, useful for keeping indentation of code (possibly needs columns=flexible)
  keywordstyle=\color{keywordsColor}\bfseries,       % keyword style
  language=Python,                 % the language of the code (can be overrided per snippet)
  otherkeywords={*,function, Seq, add,empty},           % if you want to add more keywords to the set
  numbers=left,                    % where to put the line-numbers; possible values are (none, left, right)
  numbersep=5pt,                   % how far the line-numbers are from the code
  numberstyle=\tiny\color{commentsColor}, % the style that is used for the line-numbers
  rulecolor=\color{black},         % if not set, the frame-color may be changed on line-breaks within not-black text (e.g. comments (green here))
  showspaces=false,                % show spaces everywhere adding particular underscores; it overrides 'showstringspaces'
  showstringspaces=false,          % underline spaces within strings only
  showtabs=false,                  % show tabs within strings adding particular underscores
  stepnumber=1,                    % the step between two line-numbers. If it's 1, each line will be numbered
  stringstyle=\color{stringColor}, % string literal style
  tabsize=2,	                   % sets default tabsize to 2 spaces
  title=\lstname,                  % show the filename of files included with \lstinputlisting; also try caption instead of title
  columns=fixed                    % Using fixed column width (for e.g. nice alignment)
}

\begin{lstlisting}[frame=single,mathescape, caption={Sliding Window Heuristic with Selection of First Service from Optimal Combination},label={lst:slidingwindowfirstservice}]
  function SlidingWindowHeuristic(verticesList, w){
      $\text{G'}$ = []
      for i from 0 to length(verticesList) - w + 1
      {
          minMetric = $\infty$
          minMetricCombination = []
          for windowIndex from i to i + w - 1{
              currentCombination = verticesList[windowIndex].services
              totalMetric = 0
              for service in currentCombination{
                  totalMetric += M(service)
              }
              currentMetric = totalMetric / length(currentCombination)
              if currentMetric < minMetric{
                  minMetric = currentMetric
                  minMetricCombination = currentCombination
              }
          }
          if isLastWindowFrame(){
            $\text{G'}$.append(minMetricCombination)
          }else{
              if length(minMetricCombination) > 0
              $\text{G'}$.append(minMetricCombination[0])
          }
      }
      return  $\text{G'}$
  }
  \end{lstlisting}

  The function SlidingWindowHeuristic systematically processes a list of vertices, each associated with a list of services, to identify optimal service combinations using a sliding window approach, given the constraints set by parameters verticesList and w (window size).

  Initially, the function establishes  $\text{G'}$ to store the optimal services or combinations identified during the process (line 2). It iterates from the start to the feasible end of the vertex list to ensure each possible window of services is evaluated (line 3). For each window, the function initializes minMetric to infinity and an empty list minMetricCombination to store the best service combination found within that specific window (line 5-6).

  Within each window, the function iterates through the vertices (line 7), calculating the total metric for services in the current vertex (lines 10-12). It then determines the average metric for these services and checks if this average is the lowest encountered so far within the current window (lines 13-15). If so, it updates minMetric and records the current combination as the best for this window.

  The function then checks if it is processing the last window frame using isLastWindowFrame() (line 19). If true, all services in the best combination for this window are added to the result list; otherwise, only the first service of the best combination is selected (lines 20-23). The function concludes by returning the  $\text{G'}$ (line 26), which contains the selected services or combinations based on the heuristic evaluation across all windows.
%\AG{It is imperative to bear in mind that the merging operations subsequent to the selection process and the joining operations subsequent to the branching process are executed with distinct objectives. In the former case, the primary aim is to optimize quality, whereas in the latter, the foremost objective is to minimize it.}
