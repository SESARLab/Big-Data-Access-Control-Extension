\section{Maximizing the Pipeline Instance Quality}\label{sec:heuristics}
%
% %Ovviamente non è sufficiente scegliere il best service per ogni vertice, ma diventa un problema complesso dove si devono calcolare/valutare tutte le possibili combinazioni dei servizi disponibili, tra le quali scegliere la migliore.
Our goal is to generate a pipeline instance with maximum quality, which addresses data protection requirements with the minimum amount of information loss across the pipeline execution. To this aim, we first discuss the role of some metrics (\cref{sec:metrics}) to specify and measure data quality, and describe the ones used in the paper.
Then, we prove that the problem of generating a pipeline instance with maximum quality is NP-hard (\cref{sec:nphard}). Finally, we present a parametric heuristic (\cref{subsec:heuristics}) tailored to address the computational complexity associated with enumerating all possible combinations within a given set. The primary aim of the heuristic is to approximate the optimal path for service interactions and transformations, particularly within the landscape of more complex pipelines composed of numerous nodes and candidate services. Our focus extends beyond identifying optimal combinations, encompassing an understanding of the quality changes introduced during the transformation processes.

%Inspired by existing literature, these metrics, categorized as quantitative and statistical, play a pivotal role in quantifying the impact of policy-driven transformations on the original dataset.

\subsection{Quality Metrics}\label{subsec:metrics}
%Ensuring data quality is mandatory to implement data pipelines that provide accurate results and decision-making along the whole pipeline execution. To this aim, we define two metrics evaluating the quality loss introduced by our policy-driven transformation in Section~\cite{ADD} on the input dataset \origdataset at each step of the data pipeline. Our metrics can be classified as \emph{quantitative} and \emph{qualitative}~\cite{ADD}, and compare the input dataset \origdataset\ and dataset \transdataset\ generated by enforcing data protection requirements on \origdataset.
Ensuring data quality is mandatory to implement data pipelines that provide accurate results and decision-making along the whole pipeline execution. To this aim, quality metrics evaluate the quality loss introduced at each step of the data pipeline, and can be classified as \emph{quantitative} or \emph{qualitative}~\cite{ADD}.
Quantitative metrics monitor the amount of data lost during data transformations as the quality difference between datasets \origdataset\ and \transdataset.
Qualitative metrics evaluate changes in the properties of datasets \origdataset\ and \transdataset. For instance, qualitative metrics can measure the changes in the statistical distribution of the two datasets.

In this paper, we provide two metrics, one quantitative and one qualitative, that compare the input dataset \origdataset\ and dataset \transdataset\ generated by enforcing data protection requirements (i.e., our policy-driven transformation in Section~\cite{ADD}) on \origdataset\ at each step of the data pipeline.

\subsubsection{Jaccard coefficient}
The Jaccard coefficient is a quantitative metric that can be used to measure the difference between the elements in two datasets.
It is defined as:\[J(X,Y) = \frac{|X \cap Y|}{|X \cup Y|}\]
where X and Y are two datasets of the same size.

The Jaccard coefficient is computed by dividing the cardinality of the intersection of two sets by the cardinality of their union. It ranges from 0 to 1, where 0 indicates no similarity and 1 indicates complete similarity between the datasets.

The Jaccard coefficient has several advantages. Unlike other similarity measures, such as Euclidean distance, it is not affected by the magnitude of the values in the dataset. It is suitable for datasets with categorical variables or nominal data, where the values do not have a meaningful numerical interpretation.


\subsubsection{Jensen-Shannon Divergence}
The Jensen-Shannon divergence (JSD) is a quantitative metric that can be used to measure the dissimilarity between the probability distributions of two datasets. It is a symmetrized version of the KL divergence~\cite{ADD} .

The JSD between X and Y  is defined as:

\[JSD(X, Y) = \frac{1}{2} \left( KL(X || M)
  + KL(Y || M) \right)\]

where X and Y are two datasets of the same size, and M$=$0.5*(X+Y) is the average distribution.

JSD incorporates both the KL divergence from X to M and from Y to M. It provides a balanced measure of dissimilarity that is symmetric and accounts for the contribution from both datasets.
%
JSD can compare the dissimilarity of the two datasets, providing a symmetric and normalized measure that considers the overall data distribution.
%
It provides a more comprehensive understanding of the dissimilarity between X and Y, taking into account the characteristics of both datasets.

\vspace{0.5em}

We note that our metrics can be applied either to the entire dataset or to specific features only. The features can be assigned with equal or varying importance, providing a weighted version of the metrics, thus enabling the prioritization of important features that might be possibly lost during the policy-driven transformation in Section~\cite{ADD}. A complete taxonomy of possible metrics is however outside the scope of this paper and will be the target of our future work.

\subsection{NP-Hardness of the Max Quality Pipeline Instantiation Process}\label{sec:nphard}
\hl{se lo definiamo in maniera formale come il problema di trovare un'istanza valida in accordo alla definizione di istanza tale che non ne esiste una con un loss piu' piccolo?}

\begin{definition}[Max Quality Pipeline Instantiation Process]\label{def:MaXQualityInstance}
  Given \textit{dtloss}$_i$ the value of the quality metric computed after applying the transformation of the policy matching the service selected to instantiate vertex  \vi{i}$\in$$\V_S$, the Max quality \problem is the case in which the \emph{pipeline instantiation} function returns a \pipelineInstance where the \textit{dtloss}$_i$ sum is maximized.
\end{definition}

The Max Quality \problem is a combinatorial selection problem and is NP-hard, as stated by Theorem \ref{theorem:NP}. However, while the overall problem is NP-hard, there is a component of the problem that is solvable in polynomial time: matching the profile of each service with the node policy. This can be done by iterating over each node and each service, checking if the service matches the node’s policy. This process would take $O(|N|*|S|)$ time. This is polynomial time complexity.

\begin{theorem}\label{theorem:NP}
  The Max Quality  \problem is NP-Hard.
\end{theorem}
\emph{Proof: }
The proof is a reduction from the multiple-choice knapsack problem (MCKP), a classified NP-hard combinatorial optimization problem, which is a generalization of the simple knapsack problem (KP) \cite{}. In the MCKP problem, there are $t$ mutually disjoint classes $N_1,N_2,\ldots,N_t$ of items to pack in some knapsack of capacity $C$, class $N_i$ having size $n_i$. Each item $j$$\in$$N_i$ has a profit $p_{ij}$ and a weight $w_{ij}$; the problem is to choose one item from each class such that the profit sum is maximized without having the weight sum to exceed C.

    The MCKP can be reduced to the Max quality \problem in polynomial time, with $N_1,N_2,\ldots,N_t$ corresponding to $S^c_{1}, S^c_{1}, \ldots, S^c_{u},$, $t$$=$$u$ and $n_i$ the size of $S^c_{i}$. The profit $p_{ij}$ of item $j$$\in$$N_i$ corresponds to \textit{dtloss}$_{ij}$ computed for each candidate service $s_j$$\in$$S^c_{i}$, while $w_{ij}$ is uniformly 1 (thus, C is always equal to the cardinality of $V_C$).

    Since the reduction can be done in polynomial time, our problem is also NP-hard. (non è sufficiente, bisogna provare che la soluzione di uno e' anche soluzione dell'altro)


    \begin{example}[Max-Quality Pipeline Instance]
      Let us consider a subset \{\vi{5}, \vi{6}, \vi{7}\} of the pipeline template \tChartFunction in \cref{sec:example}.
      Each vertex is associated with three candidate services, each having a profile. The filtering algorithm matches each candidate service's profile with the policies annotating the corresponding vertex. It returns the set of services whose profile matches a policy.

      The comparison algorithm is then applied to the set of services $S'_*$ and it returns a ranking of the services.
      The ranking is based on the amount of data that is anonymized by the service.
      The ranking is listed in \cref{tab:instance_example_maxquality} and it is based on the transformation function of the policies,
      assuming that a more restrictive transformation function anonymizes more data affecting negatively the position in the ranking.
      For example, \s{11} is ranked first because it anonymizes less data than \s{12} and \s{13}.
      The ranking of \s{22} and \s{23} is based on the same logic.
      Finally, the ranking of \s{31}, \s{32} is influenced by the environment state at the time of the ranking.
      For example, if the environment in which the visualization is performed is a CT facility, then \s{31} is ranked first and \s{32} second;
      thus because the facility is considered a less risky environment than the cloud.

    \end{example}

    % The metrics established will enable the quantification of data loss pre- and post-transformations.
    % In the event of multiple service interactions, each with its respective transformation,
    % efforts will be made to minimize the loss of information while upholding privacy and security standards.
    % Due to the exponential increase in complexity as the number of services and transformations grow,
    % identifying the optimal path is inherently an NP-hard problem.
    % As such, we propose some heuristics to approximate the optimal path as closely as possible.
    %To evaluate their efficacy, the heuristically generated paths will be compared against the optimal solution.

    \subsection{Heuristic}\label{subsec:heuristics}
    %The computational challenge posed by the enumeration of all possible combinations within a given set is a well-established NP-hard problem.}
    %The exhaustive exploration of such combinations swiftly becomes impractical in terms of computational time and resources, particularly when dealing with the analysis of complex pipelines.
    %In response to this computational complexity, the incorporation of heuristic emerges as a strategy to try to efficiently address the problem.
    \hl{HO RIVISTO IL PARAGRAFO VELOCEMENTE GIUSTO PER DARE UN'INDICAZIONE. DOBBIAMO USARE LA FORMALIZZAZIONE E MAGARI FORMALIZZARE ANCHE LO PSEUDOCODICE.} We design and implement a heuristic algorithm for computing the pipeline instance maximizing data quality. Our heuristic is built on a \emph{sliding window} and aims to minimize information loss according to quality metrics. At each step, a set of nodes in the pipeline template $\tChartFunction$ is selected according to a specific window w=[i,j], where $i$ and $j$ are the starting and ending depth of window w. Service filtering and selection in Section~\ref{sec:instance} are then executed to minimize information loss in window w. The heuristic returns as output the list of services instantiating nodes at depth $i$. A new window w=[i+1,j+1] is considered until $j$+1 is equal to the max depth of $\tChartFunction$, that is the window reaches the end of the template.
%For example, in our service selection problem where the quantity of information lost needs to be minimized, the sliding window algorithm can be used to select services composition that have the lowest information loss within a fixed-size window.
This strategy ensures that only services with low information loss are selected at each step, minimizing the overall information loss. Pseudo-code for the sliding window algorithm is presented in Algorithm 1.

\lstset{ %
  backgroundcolor=\color{white},   % choose the background color; you must add \usepackage{color} or \usepackage{xcolor}
  basicstyle=\footnotesize,        % the size of the fonts that are used for the code
  breakatwhitespace=false,         % sets if automatic breaks should only happen at whitespace
  breaklines=true,                 % sets automatic line breaking
  captionpos=b,                    % sets the caption-position to bottom
  commentstyle=\color{commentsColor}\textit,    % comment style
  deletekeywords={list},            % if you want to delete keywords from the given language
  escapeinside={\%*}{*)},          % if you want to add LaTeX within your code
  extendedchars=true,              % lets you use non-ASCII characters; for 8-bits encodings only, does not work with UTF-8
  frame=tb,	                   	   % adds a frame around the code
  keepspaces=true,                 % keeps spaces in text, useful for keeping indentation of code (possibly needs columns=flexible)
  keywordstyle=\color{keywordsColor}\bfseries,       % keyword style
  language=Python,                 % the language of the code (can be overrided per snippet)
  otherkeywords={*,to,function, Seq, add,empty},           % if you want to add more keywords to the set
  numbers=left,                    % where to put the line-numbers; possible values are (none, left, right)
  numbersep=5pt,                   % how far the line-numbers are from the code
  numberstyle=\tiny\color{commentsColor}, % the style that is used for the line-numbers
  rulecolor=\color{black},         % if not set, the frame-color may be changed on line-breaks within not-black text (e.g. comments (green here))
  showspaces=false,                % show spaces everywhere adding particular underscores; it overrides 'showstringspaces'
  showstringspaces=false,          % underline spaces within strings only
  showtabs=false,                  % show tabs within strings adding particular underscores
  stepnumber=1,                    % the step between two line-numbers. If it's 1, each line will be numbered
  stringstyle=\color{stringColor}, % string literal style
  tabsize=2,	                   % sets default tabsize to 2 spaces
  title=\lstname,                  % show the filename of files included with \lstinputlisting; also try caption instead of title
  columns=fixed                    % Using fixed column width (for e.g. nice alignment)
}


\begin{lstlisting}[frame=single,mathescape, caption={Sliding Window Heuristic with Selection of First Service from Optimal Combination},label={lst:slidingwindowfirstservice}]
  selectedServices = empty
  for i from 0 to length(serviceCombinations):
      minMetricCombination = None
      minMetric = $+\infty$
      M = JSD or J //JSD or Jaccard coefficient

      for j from i to i + windowSize:
          totalMetric = 0
          for service in serviceCombinations[j]:
              totalMetric += M(service)
              currentMetric = totalMetric / length(serviceCombinations[j])
              if currentMetric < minMetric:
                  minMetric = currentMetric
                  minMetricCombination = serviceCombinations[j]
                  firstService = serviceCombinations[j][0]
      add firstService to instance
  return instance

  \end{lstlisting}

The pseudocode implemets function {\em SlidingWindowHeuristic}, which takes a sequence of nodes and a window size as input and returns a set of selected nodes as output. The function starts by initializing an empty set of selected nodes (line 3). Then, for each node in the sequence (lines 4--12), the algorithm iterates over the nodes in the window (lines 7--11) and selects the node with the lowest metric value (lines 9-11). The selected node is then added to the set of selected nodes (line 12). Finally, the set of selected nodes is returned as output (line 13).

We note that a window of size 1 corresponds to the \emph{greedy} approach, while a window of size N, where N represents the total number of nodes, corresponds to the \emph{exhaustive} method.

The utilization of heuristic in service selection can be enhanced through the incorporation of techniques derived from other algorithms, such as Ant Colony Optimization or Tabu Search.
By integrating these approaches, it becomes feasible to achieve a more effective and efficient selection of services, with a specific focus on eliminating paths that have previously been deemed unfavorable.

%\AG{It is imperative to bear in mind that the merging operations subsequent to the selection process and the joining operations subsequent to the branching process are executed with distinct objectives. In the former case, the primary aim is to optimize quality, whereas in the latter, the foremost objective is to minimize it.}
